\documentclass[
    12pt,                % Schriftgroesse 12pt
    a4paper,             % Layout fuer Din A4
    ngerman, 			% deutsche Sprache, global
    %oneside,            % einseitig
    %twoside,             % Layout fuer beidseitigen Druck
    openright,          % Start auf rechter Seite   
    headinclude,         % Kopfzeile wird Seiten-Layouts mit beruecksichtigt
    headsepline,         % horizontale Linie unter Kolumnentitel
	  %footinclude,
	  %footexclude,		%fusszeile nicht bercksichtigen
    %plainheadsepline,    % horizontale Linie auch beim plain-Style
    BCOR12mm,            % Korrektur fuer die Bindung
    DIV14,               % DIV-Wert fuer die Erstellung des Satzspiegels, siehe scrguide
    %DIVcalc,							% automatische Berechnung des Satzspiegels
    halfparskip,         % Absatzabstand statt Absatzeinzug
    %openany,             % Kapitel knnen auf geraden und ungeraden Seiten beginnen
    %openright,							% Kapitel auf rechten Seiten beginnen!
    bibtotoc,              % Literaturverz. wird ins Inhaltsverzeichnis eingetragen
    liststotoc,						% Verzeichnisse ins Inhaltsverzeichnis
    %pointlessnumbers,    % Kapitelnummern immer ohne Punkt
    tablecaptionabove,   % korrekte Abstaende bei TabellenUEBERschriften
    fleqn,               % fleqn: Glgen links (statt mittig)
    svgnames,           % colornames
    %draft,               % Keine Bilder in der Anzeige, overfull hboxes werden angezeigt
    appendixprefix				%berschriften des Anhangs +"Anhang"
    %chapterprefix,				%berschriften der Kapitel +"KAPITEL"
    %tablecaptionabove,    %tabellen immer als berschriften
    %abstracton,
    %pdftex						%berschrift Zusammenfassung einschalten
     ]{scrreprt}

% Formatierungen
\usepackage{typearea}
\areaset[1cm]{16cm}{26cm}
\usepackage[small,bf,up]{caption}  %nicer captions
\renewcommand{\captionfont}{\small}
\usepackage[hyphens]{url}

% Für PDF-Erstellung
\usepackage{ae}                 
% schönere Symbole
\usepackage{latexsym}			

\usepackage{color}
\usepackage{float}
\usepackage{rotating}

% Codehilighting
\usepackage{minted}
\usemintedstyle{murphy}
% Floating figures
\usepackage{wrapfig}

% Quellcode
% PseudoCode
\usepackage[chapter]{algorithm}
\usepackage{algpseudocode}
%\usepackage{algorithmicx}
\floatname{algorithm}{Algorithmus}
\algrenewcommand{\algorithmiccomment}[1]{\hskip1em\textcolor{gray!60}{$\rhd$ #1}}
\renewcommand{\listalgorithmname}{Algorithmen}
\def\algorithmautorefname{Algorithmus}



% Zeichensätze
\usepackage[utf8]{inputenc}
\usepackage[T1]{fontenc} % T1-kodierte Schriften, korrekte Trennmuster fuer Worte mit Umlauten
\usepackage{lmodern}

   
% Hyperref
\usepackage{hyperref}
\definecolor{darkblue}{rgb}{0,.05,.54}
\definecolor{darkgreen}{rgb}{0,.54,.05}
\definecolor{almostblack}{rgb}{.1,.1,.1}
\makeatletter
\hypersetup{pdftitle={\@title}, pdfauthor={\@author}, pdfsubject={\@subject}, colorlinks=true, breaklinks=false, linkcolor=almostblack, menucolor=darkblue, urlcolor=darkblue, citecolor=darkblue, filecolor=darkblue, pdfborder=0 0 0}
\makeatother
\urlstyle{same}
\usepackage[all]{hypcap}

% Sprache
\usepackage[german,ngerman]{babel}

% Mathe und Formeln
\usepackage{calc}
\usepackage[centertags]{amsmath}
\usepackage{amssymb,amsthm,amsfonts}
\usepackage{cancel}  %%druchstreichen von Formeln

% Programmieren mit Latex
\usepackage{ifthen}

% Zeilenabstand
\usepackage{setspace}
\setstretch{1.1}

%setzen von baumstrukturen
\usepackage{dirtree}   
% WINDINGS additional Symbols
\usepackage{bbding}    

% kleinere Anstände bei enums und itemizes möglich
\usepackage{paralist}      

% Fuer anspruchsvolle Tabellen 
\usepackage{longtable, colortbl}
\usepackage{multicol, multirow}

% Für Grafiken
\usepackage{tikz}
\usepackage{pgfplots}
\usetikzlibrary{arrows,shapes,fit,positioning,backgrounds,shadows}

% Zur Darstellung des Euro-Symbols
\usepackage{eurosym, wasysym}
\selectlanguage{german}
% Fuer @author == \theauthor
\usepackage{titling}
% Fuer Bibtex nach APA Style (American Psychology Association)   %%
\usepackage[numbers]{natbib}

% intoc zur Aufnhame des Abkuerzungs- und Symbolverzeichnisses ins Inhaltsverzeichnis  
\usepackage[intoc]{nomencl}
\setlength{\nomlabelwidth}{.20\hsize}
%\renewcommand{\nomlabel}[1]{#1 \dotfill}
\setlength{\nomitemsep}{-\parsep}
\makenomenclature
\renewcommand{\nomname}{Abk\"urzungsverzeichnis}
\newcommand{\nomaltname}{Symbolverzeichnis}
\newcommand{\nomaltpreamble}{}
\newcommand{\nomaltpostamble}{}
\newcommand{\usetwonomenclatures}{\nomenclature[\switchnomitem]{}{}}
\newcommand{\switchnomitem}{R}
\renewcommand{\nomgroup}[1]{%
\ifthenelse{\equal{#1}{\switchnomitem}}{\switchnomalt}{}}
\newcommand{\switchnomalt}{%
\end{thenomenclature}
\newpage
\renewcommand{\nomname}{\nomaltname}
\renewcommand{\nompreamble}{\nomaltpreamble}
\renewcommand{\nompostamble}{\nomaltpostamble}
\begin{thenomenclature}
}


% Stichwortverzeichnis
\usepackage{makeidx}
\makeindex

% Hervorhebung der Abkuerzungsbuchstaben   %%
\usepackage[normalem]{ulem}
\newcommand{\m}[1]{\uline{#1}}

% Code-Hervorhebung

\usepackage{caption}
\DeclareCaptionFormat{listing}{\par\hrule #1#2#3}
\captionsetup[lstlisting]{format=listing, singlelinecheck=false, margin=0pt, font={bf,footnotesize}}
\captionsetup[figure]{format=listing, singlelinecheck=false, margin=60pt, font={it,footnotesize,centering}}

\parskip 12pt


\renewcommand{\listoflistingscaption}{Quellcodes}
\renewcommand{\listoflistingscaption}{Quellcode}
%%%%%%%%%%%%%%%%%%%%%%%%%%%%%%%%%%%%%%%%%%%%%%%%%%%%%%%%%%%%%%%%%%%%%%%%%%%%%%%%%%%%%%%%%%%
%%                                   COMMAND SETUP                                       %%
%%%%%%%%%%%%%%%%%%%%%%%%%%%%%%%%%%%%%%%%%%%%%%%%%%%%%%%%%%%%%%%%%%%%%%%%%%%%%%%%%%%%%%%%%%%
\newcommand{\HRule}{\rule{\linewidth}{0.5mm}}

%#1 Breite
%#2 Datei (liegt im image Verzeichnis)
%#3 Beschriftung
%#4 Label fuer Referenzierung
\newcommand{\image}[4]{
\begin{figure}[H]
\centering
\includegraphics[width=#1]{contents/image/#2}
\caption{#3}
\label{#4}
\end{figure}
}

%#1 Datei (liegt im graphic Verzeichnis)
%#2 Beschriftung
%#3 Label fuer Referenzierung
\newcommand{\tikzimage}[3]{%
\begin{figure}[H]%
\centering%
\input{contents/graphic/#1.tikz}%
\caption{#2}%
\label{#3}%
\end{figure}
}

%#1 Datei (liegt im graphic Verzeichnis)
%#2 Beschriftung
%#3 Label fuer Referenzierung
%#4 Skalierungsfaktor
\newcommand{\scaletikzimage}[4]{%
\begin{figure}[H]%
\centering%
\scalebox{#4}{%
\input{contents/graphic/#1.tikz}}%
\caption{#2}%
\label{#3}%
\end{figure}
}

%#1 Breite
%#2 Höhe
%#2 Datei (liegt im image Verzeichnis)
%#3 Beschriftung
%#4 Label fuer Referenzierung
\newcommand{\imagebh}[5]{
\begin{figure}[H]
\centering
\includegraphics[width=#1, height=#2]{contents/image/#3}
\caption{#4}
\label{#5}
\end{figure}
}

%#1 Breite
%#2 Datei (liegt im image Verzeichnis)
%#3 zugehörige Bildunterschrift
%#4 Beschriftung
%#5 Label fuer Referenzierung
\newcommand{\mathimage}[5]{
\begin{figure}[H]
\centering
\includegraphics[width=#1]{contents/image/#2}\\
#3
\caption{#4}
\label{#5}
\end{figure}
}
%#1 algorithm name
%#2 algorithm label
%#3 file name in code-folder
\newcommand{\pseudo}[3]{%
\small%
\begin{algorithm}[H]%
\caption{#1}%
\label{#2}%
\input{contents/code/#3.tex}%
\end{algorithm}%
\normalsize%
}
%#1 algorithm name
%#2 algorithm label
%#3 file name in code-folder
\newcommand{\realcode}[3]{%
\definecolor{bg}{rgb}{0.95,0.95,0.95}
\small%
\begin{listing}[H]%
\input{contents/code/#3.tex}%
\caption{#1}
\label{#2}%
\end{listing}%
\normalsize%
}

%#1 Text der als todo dargestellt werden soll
\newcommand{\todo}[1]{
\begin{quote}
\textcolor{red}{\textbf{TODO: #1}}
\end{quote}
}

\newcommand{\rimage}[2]{
\begin{figure}[H]
\centering
#1
%\caption{#2}
\end{figure}
}

\newcommand \rack {
       {\LARGE $\square$}
}

\newcommand \deftab
{\hspace{1.5cm}\=abcdfffefghijk\hspace{1cm}\=1\hspace{1.5cm}\=1\hspace{1.5cm}\=1\hspace{1.5cm}\=1\hspace{1.5cm}\=1\hspace{1.5cm}\=asdjadj\kill}

\newcommand \einsbisfuenf
{\> {\bf -2} \> {\bf -1} \> {\bf 0} \> {\bf 1} \> {\bf 2} \>}

% #1 videofile
% #2 scalefactor
\newcommand{\video}[2]{%
\includemovie[text={\includegraphics[scale=#2]{praesi/video/#1.png}}, autoplay, mouse=true, repeat=1]{}{}{praesi/video/#1.swf}}

% The \munepsfig command is used to insert a new EPS figure 
% into our document.  Usage is:
%
%	\munepsfig[args]{filename}{caption}
%
% where:
%	- the optional 'args' argument is passed to the
%	  embedded \includegraphics command, this can be used
%	  to scale the figure or rotate it.
%	- 'filename' is the name of the EPS file in the 'figures'
%	  directory that is to be inserted (note that 'filename'
%         should not include the '.eps' extension).
%	- 'filename' also serves as the label for the figure.
%	  with the text 'fig:' prepended.
%
% Sample Usage:
% 	\munepsfig[scale=0.5,angle=90]{barchart}{Population over time}

% inserts the EPS file 'figures/barchart.eps' reduced in size by 50%
% rotated 90 degrees and with the caption "Popuation over Time."
% We can refer to that figure as Figure~\ref{fig:barchart} in the text.
%
\newcommand{\munepsfig}[3][scale=1.0]{%
	\begin{figure}[!htbp]
		\centering
		\vspace{2mm}
		\includegraphics[#1]{figures/#2}
		\caption{#3}
		\label{fig:#2}
	\end{figure}
}
\newcommand{\muneps}[3][scale=1.0]{%
		\vspace{2mm}
		\includegraphics[#1]{figures/#2}
		\caption{#3}
		\label{fig:#2}
}
% \munlepsfig command inserts a figure in landscape mode.  The
% entire page is rotated to accommodate the figure.  The arguments
% are the same as for \munepsfig, above
%
\newcommand{\munlepsfig}[3][scale=1.0]{%
	\begin{sidewaysfigure}[!htbp]
		\centering
		\vspace{2mm}
		\includegraphics[#1]{figures/#2.eps}
		\caption{#3}
		\label{fig:#2}
	\end{sidewaysfigure}
}

% The 'muntxtfig' environment is used to insert 'textual' figures
% into our document, such as brief source code snippets.  Usage is:
%
%	\begin{muntxtfig}[spacing]{label}{caption}{width}
%	  text for 'figure'
%	\end{muntxtfig}
%
% where:
%	- 'spacing' is a number representing the baselinestretch
%	  (line-spacing) to use for the text figure.  Default is
%          single-spacing.
%	- 'label' is the label to be used for referencing.
%	   The figure can be referenced as Figure~\ref{fig:label}.
%	- 'caption' is the caption to display below the figure.
%	- 'width' is the width of the minipage in which the text figure
%	  is formatted.
%
% Sample usage:
% \begin{muntxtfig}[1.0]{code}{Hello World}{0.5\textwidth}
%   Some text
% \end{muntxtfig}
%
\newenvironment{muntxtfig}[4][\spacing]{%
	\begin{figure}[!htbp]
		\centering
		\def\muncaption{#3}%
		\def\munlabel{#2}%
		\renewcommand{\baselinestretch}{#1}%
		\normalsize%
		\begin{minipage}{#4}
		\hrule \hrule
		\bigskip
}{%
		\hrule \hrule
		\end{minipage}
		\renewcommand{\baselinestretch}{\spacing}%
		\normalsize%
		\caption{\muncaption}
		\label{fig:\munlabel}
	\end{figure}
}

% The 'muntab' environment is used to insert a new table into our document.
% Usage is:
%
%	\begin{muntab}{table_format}{label}{caption}
%        table contents
%       \end{muntab}
%
% where:
%	- 'label' is the label used to reference the Table.
%	   We can refer to the table as Table~\ref{tab:label} in
%          the text.
%	- 'caption' is the caption placed above the Table.
%
%
\newenvironment{muntab}[3]{%
	\begin{table}[!htbp]
		\centering
		\caption{#3}
		\label{tab:#2}
		\vspace{2mm}
		\begin{tabular}{#1}
}{%
		\end{tabular}
	\end{table}
}

% The 'munltab' environment is like muntab, but displays the table
% in landscape mode on its own page.  See muntab environment for usage.
%
\newenvironment{munltab}[3]{%
	\begin{sidewaystable}
		\centering
		\caption{#3}
		\label{tab:#2}
		\vspace{2mm}
		\begin{tabular}{#1}
}{%
		\end{tabular}
	\end{sidewaystable}
}

% The \muneqn environment is used to add an equation to the thesis.
% Usage is:
%
%	\begin{muneqn}{label}
%         equation body
%	\end{muneqn}
%
% where:
%	- 'label' is the label used to reference the Equation.
%	   We can refer to the equation as Table~\ref{eqn:label} in
%          the text.
%
\newenvironment{muneqn}[1]{%
	\begin{equation}
		\label{eqn:#1}
}{
	\end{equation}
}


%%%%%%%%%%%%%%%%%%%%%%%%%%%%%%%%%%%%%%%%%%%%%%%%%%%%%%%%%%%%%%%%%%%%%%%%%%%%%%%%%%%%%%%%%%%
%%                                 Selbständigkeitserklärung                             %%
%%%%%%%%%%%%%%%%%%%%%%%%%%%%%%%%%%%%%%%%%%%%%%%%%%%%%%%%%%%%%%%%%%%%%%%%%%%%%%%%%%%%%%%%%%%
\makeatletter
\newcommand{\erklaerung}{
\newpage
\section*{Eidesstattliche Erklärung}
\vspace{25mm}

Ich erkläre hiermit gemäß § 17 Abs. 2 APO, dass ich die vorstehende \degree arbeit selbständig verfasst und keine anderen als die angegebenen Quellen und Hilfsmittel benutzt habe.\\[20mm]

\begin{minipage}{0.4\textwidth}
\location , \@date \hfill \\
\textcolor{white}{M} 
\end{minipage}
\begin{minipage}{0.6\textwidth}
\begin{flushright}
\begin{center}
\textcolor{white}{M}\ldots\ldots\ldots\ldots\ldots\ldots\ldots\ldots\ldots\ldots\ldots\ldots\\
\@author \vfill
\end{center}
\end{flushright}
\end{minipage}
\newpage
}
\makeatother


% Meta-Informationen
\author{Autor}
\title{Titel (deutsch)}
\subtitle{Title (english)}
\newcommand{\degree}{Abschluss}
\newcommand{\assignment}{Assignment}
\newcommand{\studycourse}{Studiengang}
\newcommand{\advisor}{Betreuer}
\newcommand{\location}{Bamberg}
\subject{\degree arbeit im Studiengang \studycourse\ der Fakultät Wirtschaftsinformatik und Angewandte Informatik der Otto-Friedrich-Universität Bamberg}
\date{xx.xx.xxxx}

